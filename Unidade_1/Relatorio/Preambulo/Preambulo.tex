\documentclass[12pt,a4paper]{article}
\usepackage[utf8]{inputenc}
\usepackage[portuguese, brazilian]{babel} %hifenização
\usepackage[T1]{fontenc} %acentuação
\usepackage{amsmath}
\usepackage{amsfonts}
\usepackage{amssymb}
\usepackage{graphicx}

\usepackage{color}
\usepackage[dvipsnames,table]{xcolor}

\definecolor{lightgray}{gray}{0.9}

\usepackage{multirow}

%Para que as referencias bibliográficas apareçam no sumario 
\usepackage[nottoc]{tocbibind}


%Margens do documento
\usepackage{geometry}
\geometry{hdivide={2cm,*,2cm},vdivide={2cm,*,2cm}}

%Texto automatico
\usepackage{lipsum}


%Cria hiperlinjs no documento
\usepackage[pdftex,hidelinks
]{hyperref}
\hypersetup{
%	colorlinks=true,
%	linkcolor=red,
	%citecolor=green,
	%filecolor=magenta,
%	urlcolor=cyan,
}

%Formatação da fonte da legenda
\usepackage[font={small},labelfont={bf}, margin=1cm]{caption}

%Identar o primeiro paragrafo
\usepackage{indentfirst}

%Espacamento entre linhas
\usepackage{setspace} 

%Espaçamento entre Parágrafos
\setlength{\parskip}{0.5cm}



\newcommand{\fuzzy}{\textit{Fuzzy }}
